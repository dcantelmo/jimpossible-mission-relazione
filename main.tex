% Basic LaTeX boilerplate for a report/article
\documentclass[11pt,a4paper]{article}

% Language and encoding
\usepackage[utf8]{inputenc}
\usepackage[T1]{fontenc}
\usepackage[italian]{babel}

% Common packages
\usepackage{lmodern}
\usepackage{geometry}
\usepackage{graphicx}
\usepackage{amsmath, amssymb}
\usepackage{hyperref}
\usepackage{csquotes}
\usepackage{booktabs}
\usepackage{microtype}
\usepackage{xcolor}
\usepackage{tikz}
\usepackage{forest}
\usepackage{listings}
\lstset{
    language=Java,
    basicstyle=\ttfamily\small,
    numbers=left,
    numberstyle=\tiny,
    frame=single,
    breaklines=true,
    keywordstyle=\color{blue},
    commentstyle=\color{gray},
    stringstyle=\color{red}
}

% Page geometry
\geometry{margin=2.5cm}

% Metadata
\title{{\Huge Progetto: JImpossible Mission}}
\author{Daniele Cantelmo\\Matricola 1815100\\Teledidattica}
\date{Corso di Metodologie di Programmazione\\\today}

\begin{document}
\maketitle
\begin{figure}[h]
    \centering
    \includegraphics[width=0.8\textwidth]{assets/main.jpg}
\end{figure}
\newpage
\tableofcontents
\newpage

\section{Introduzione}
Il documento descrive il progetto \textbf{JImpossible Mission}, un progetto didattico che consiste nella realizzazione di un videogioco
in Java basata sul classico gioco per Commodore 64 "Impossible Mission". 
\newline
\newline
Il videogioco è un platform 2D in cui il giocatore controlla un agente segreto infiltrato in una base nemica. 
Lo scopo del giocatore è quello di cercare e raccogliere, all'interno della base, tutti i frammenti di schede perforate disperse
e nascoste all'interno degli oggetti discena. Una volta raccolti tutti i frammenti sarà possibile decifrare il codice di accesso alla 
stanza di controllo e terminare la missione una volta raggiunta la porta.
\newline
\newline
Il gioco presenta varie meccaniche fondamentali, che definiscono il modo in cui il giocatore interagisce con l'ambiente di gioco e le sfide che deve affrontare.
Tra queste meccaniche principali troviamo:
\begin{itemize}
    \item Controllo del personaggio principale tramite tastiera
    \item Nemici con IA semplice
    \item Interazione con oggetti collezionabili e dispositivi (PC, ascensori)
    \item Gestione delle collisioni tra entità
    \item Ascensori
    \item Suoni e animazioni
\end{itemize}
Il gioco è stato sviluppato utilizzando Java, con versione di riferimento JDK 22.
Per la gestione dei menù, della grafica e delle animazioni è stata utilizzata la libreria Java Swing

I requisiti funzionali del progetto riguardano principalmente:
\begin{itemize}
    \item Impiego del pattern strutturale MVC (Model-View-Controller)
    \item Mondo di gioco dinamico, con entità in movimento, animazioni e suoni
    \item Ostacoli e nemici da evitare
    \item Gestione dei profili utente e persistenza dei punteggi
    \item Oggetti da raccogliere e utilizzare (Freeze, Reset ascensori ecc.)
\end{itemize}

Il gioco è stato sviluppato con l'idea di sfruttare e adottare svariati pattern di progettazione del 
software per poterne conoscere punti di forza e limiti, oltre che per rendere il codice più lineare e manutenibile.
Inoltre è stata posta particolare attenzione alla separazione delle responsabilità tra le varie componenti del pattern MVC ed è stata
promossa la modularità del codice che ha facilitato l'estensione del progetto con funzionalità aggiuntive.

\subsection{Panoramica dell'architettura}
L'architettura di \textbf{JImpossible Mission} è organizzata come segue:
\begin{itemize}
    \item \textbf{Model}: è il cuore del progetto. Gestisce le logiche del gioco, le interazioni tra le entità e rappresenta lo stato del mondo di gioco. \\[2pt]
    \textit{Subpackages}: \texttt{entity}, \texttt{manager}, \texttt{physics}, \texttt{game}, \texttt{scene}, \texttt{item}, \texttt{profile}.
    \item \textbf{View}: si occupa della rappresentazione grafica utilizzando coma base \textbf{Java Swing}, estendendone i componenti. Gestisce le interfacce dei menù, le animazioni e gli effetti audio del gioco \\[2pt]
    \textit{Subpackages}: \texttt{animation}, \texttt{audio}, \texttt{color}, \texttt{hud}, \texttt{item}, \texttt{menu}, \texttt{ui}.
    \item \textbf{Controller}: riceve gli input da tastiera (tramite \texttt{InputController}), li normalizza in comandi di gioco (azioni) e li passa al Model. Contiene inoltre il \textbf{game loop}, che scandisce l'aggiornamento periodico dello stato del gioco.
\end{itemize}

\subsection{Principali componenti}
A livello di package, l'architettura è organizzata in:
\begin{itemize}
    \item \textbf{Controller}: gestione del ciclo di gioco, input da tastiera, cambio scena e coordinamento tra Model e View.
    \item \textbf{Model}: definizione delle entità (player, nemici, oggetti, ascensori), gestione delle collisioni e delle interazioni, logica dei nemici e del punteggio, timer di gioco.
    \item \textbf{View}: pannelli grafici Swing, rendering delle scene, gestione delle animazioni e dell'HUD.
    \item \textbf{Audio / Risorse}: caricamento e riproduzione dei suoni, gestione delle sprite e delle immagini.
    \item \textbf{Persistenza}: gestione dei profili utente e salvataggio dei punteggi su file.
\end{itemize}

\subsection{Flusso di esecuzione}
Il flusso principale del gioco è il seguente:
\begin{enumerate}
    \item Il \textbf{Controller} esegue il \textit{game loop} e, ad ogni iterazione, legge lo stato corrente degli input (\texttt{InputController}).
    \item Gli input vengono trasformati in \textbf{azioni} di gioco (es. movimento a sinistra, salto, interazione) e passati al \textbf{Model}.
    \item Il \textbf{Model} aggiorna lo stato del mondo di gioco: posizione delle entità, collisioni, timer, stato dei nemici, gestione degli oggetti collezionabili.
    \item Il \textbf{Model}, estendendo \texttt{Observable}, notifica le \textbf{View}, che ridisegnano la scena corrente e l'HUD.
\end{enumerate}
Il gioco presenta le seguenti funzionalità:

\begin{itemize}
    \item Ciclo di gioco (Game loop)
    \item Motore fisico (semplice, per il player)
    \item Controllo del personaggio da tastiera
    \item Nemici con IA (semplice)
    \item Ostacoli per il personaggio
    \item Gestione delle collisioni
    \item Interazioni con oggetti (collezionabili, uso del PC, porta d'uscita, ascensori)
    \item Cambi di scena
    \item Suoni per stato e per frame
    \item Animazioni
    \item HUD
\end{itemize}

\section{Struttura del codice}
La principale metodologia adottata che ha definito l'architettura della codebase è stato il paradigma MVC (Model-View-Controller).
\newline
Ciò ha permesso di separare in maniera ben definita le componenti logiche del gioco (Model), la rappresentazione grafica (View) e la gestione degli input dell'utente (Controller).
Questo ha facilitato l'estensibilità del progetto in quanto le aree su cui apportare eventuali modifiche erano chiare e distinte e le ripercussioni su altri packages erano ben definite.

\section{Game Loop}
Per la realizzazione del progetto il primo passo compiuto è stata la creazione del game loop. 
Grazie ad esso è possibile gestire interazioni in tempo reale, aggiornare lo stato del gioco e renderizzare la grafica in modo fluido.

L'idea di partenza del game loop è quella di mantenere un ciclo continuo ed aggiornare lo stato del gioco ad intervalli regolari.
Per fare ciò si utilizza un ciclo while che continua finché il gioco è in esecuzione e si definisce un target di aggiornamenti che si vogliono effettuare.
Nel mio caso ho scelto di effettuare 60 aggiornamenti al secondo, un valore comunemente usato nei videogiochi per garantire fluidità di esecuzione senza sovraccaricare eccessivamente la CPU.

Il codice del game loop è stato implementato nel package Controller, nel seguente modo:
\newpage
\begin{lstlisting}[language=Java]
private static void runGameLoop(Model m, InputController inputController) {
        long lastTime = System.nanoTime();

        while (gameRunning && !Thread.currentThread().isInterrupted()) {
            long currentTime = System.nanoTime();
            long deltaTime = currentTime - lastTime;

            if (deltaTime >= FRAME_TIME_NS) {
                float dt = deltaTime / 1_000_000_000f;
                dt = Math.max(FRAME_TIME, Math.min(MAX_FRAME_TIME, dt));
                m.updateGameState(new HashSet<>(inputController.getInputState()), dt);
                lastTime = currentTime;
            } else {
                long remainingTime = FRAME_TIME_NS - deltaTime;
                if (remainingTime > 2_000_000) {
                    try {
                        Thread.sleep(1);
                    } catch (InterruptedException e) {
                        Thread.currentThread().interrupt();
                        break;
                    }
                }
            }
        }
    }
\end{lstlisting}

Il game loop inizia registrando il tempo corrente in nanosecondi. 
All'interno del ciclo while, calcola il tempo trascorso dall'ultimo aggiornamento (deltaTime). 
Se questo tempo supera il frame time desiderato (\texttt{FRAME\_TIME\_NS}),
aggiorna lo stato del gioco chiamando il metodo updateGameState del Model, passando gli input correnti e il delta time normalizzato. 
Dopo l'aggiornamento, aggiorna lastTime al tempo corrente. Se il deltaTime è inferiore al frame time, 
il thread dorme per un breve periodo per evitare un utilizzo eccessivo della CPU.

Ad ogni iterazione del ciclo inoltre, viene passato al Model lo stato corrente degli input raccolti dal Controller.
\newpage
\section{Gestione degli input}
Per consentire all'utente di interagire con il videogioco e controllare il personaggio, è stata definita una classe di input catturabili da tastiera.
Questa classe, chiamata InputController che cattura gli input da tastiera implementando l'interfaccia "KeyListener" e filtrando gli input per i comandi ammissibili dal Model.
Nel Model è invece definita una classe che mappa una coppia di input (tasti premuti) ad uno specifico comando di gioco, che verrà interpretato nella maniera opportuna
in base al contesto e allo stato corrente del player.
Ad esempio il tasto "A" e il tasto "Freccia Sinistra" sono entrambi associati all'azione di movimento a sinistra del personaggio.
\begin{lstlisting}[language=Java]
...
  public enum Action {
    LEFT(KeyEvent.VK_LEFT, KeyEvent.VK_A),
    RIGHT(KeyEvent.VK_RIGHT, KeyEvent.VK_D),
    UP(KeyEvent.VK_UP, KeyEvent.VK_W),
    DOWN(KeyEvent.VK_DOWN, KeyEvent.VK_S),
    ESCAPE(KeyEvent.VK_ESCAPE, KeyEvent.VK_Q),
    SELECT(KeyEvent.VK_E, KeyEvent.VK_E),
    FIRE(KeyEvent.VK_SPACE, KeyEvent.VK_ENTER),
    SKIP(KeyEvent.VK_T, KeyEvent.VK_T);
    ...
  }
\end{lstlisting}
\section{Struttura del Model}
Il Model ora riceve input normalizzati, ed essendo inserito in un game loop, può aggiornare regolarmente il proprio stato grazie al metodo updateGameState.
Questo metodo si occupa di aggiornare tutte le entità presenti nel gioco.
Per poter fare ciò ha bisogno di strutture dati di appoggio che rappresentato lo stato corrente del gioco. 
Ad alto livello, queste strutture dati sono identificate con:
\begin{itemize}
    \item Entity: oggetto base che rappresenta qualsiasi entità presente nel gioco (player, nemici, oggetti, pavimenti ecc.)
    \item EntityManager: istanza che gestisce tutte le entità
    \item Collision-Interaction-DamagerManager: istanze che gestiscono le collisioni e le interazioni tra entità
    \item GameData: oggetto che incapsula le informazioni globali del gioco quali punteggio, tempo trascorso, collezionabili trovati ecc.
    \item SceneContext: rappresenta la scena corrente del gioco, contenente le entità visibili e interagibili in quel momento.
\end{itemize}
Il model inoltre, seguendo il paradigma MVC, estende la classe Observable per notificare la View ad ogni update e mantenerla sincronizzata allo statto attuale.

\section{Posizionamento e movimento delle entità}
Ogni elemento nel gioco è rappresentato come un'istanza della classe Entity. Questa contiente informazioni riguardanti la posizione e la dimensione. 
I posizionamenti nel software sono gestiti tramite coordinate cartesiane in un sistema 2D, dove l'origine (0,0) si trova nell'angolo in alto a sinistra dello schermo.
Le coordinate sono rappresentate dalla classe Point che contiene due valori interi, x (ascissa) e y (ordinata), in pixel.

Le entità a seconda del caso, posso avere anche una velocità (rappresentata in pixel/secondo) e lo spostamento viene calcolato ad ogni aggiornamento del game loop,
utilizzando la formula:
\begin{equation}
    \text{newPosition} = \text{currentPosition} + \text{velocity} \times \Delta t
\end{equation}
dove \(\Delta t\) è il tempo trascorso dall'ultimo aggiornamento, fornito dal game loop.



\section{Risultati}
Presentare risultati, tabelle e figure.
\subsection{Esempio di tabella}
\begin{table}[h]
    \centering
    \begin{tabular}{lrr}
    \toprule
    Voce & Valore A & Valore B \\
    \midrule
    Esempio 1 & 10 & 20 \\
    Esempio 2 & 15 & 25 \\
    \bottomrule
    \end{tabular}
    \caption{Esempio di tabella.}
\end{table}

\subsection{Esempio di figura}
\begin{figure}[h]
    \centering
    % \includegraphics[width=0.7\textwidth]{path/to/image}
    \caption{Esempio di figura.}
\end{figure}

\section{Discussione}
Interpretare i risultati, limiti e lavori futuri.

\section{Conclusioni}
Riepilogo e takeaway principali.

\appendix
\section{Appendice}
Dettagli tecnici aggiuntivi.

\end{document}
